%chapter 4, Tom Mulvey, 3/7/18
\documentclass[12pt]{article}

\usepackage{amsfonts}
\usepackage{amsmath}
\usepackage{amssymb}
\usepackage{amsthm}
\usepackage{xcolor}
\usepackage{graphicx}
\usepackage{microtype}

%new macros for r,q,n,z
\newcommand{\Q}{\mathbb Q}
\newcommand{\N}{\mathbb N}
\newcommand{\Z}{\mathbb Z}
\newcommand{\R}{\mathbb R}
\newcommand{\ind}{\hspace{10mm}}

\newtheorem{thm}{Theorem}
\newtheorem{lem}[thm]{Lemma}
\newtheorem{cor}[thm]{Corollary}
\newtheorem{rem}[thm]{Remark}
\newtheorem{conj}[thm]{Conjecture}

%\fboxsep=2mm \fboxrule=1mm \fcolorbox{black}{blue!20!white}{\null}
%\fbox{\includegraphics[height=1cm]{proof.jpeg}}
%{%
%	\setlength{\fboxsep}{1pt}%
%	\setlength{\fboxrule}{2pt}%
%	\fbox{\includegraphics[height=1cm]{proof.jpeg}}
%}%

\begin{document}
    \centerline{\sc \large Tom Mulvey | Homework 5 }
    \centerline{\sc 3/27/18}

    \vspace{2pc}

	\begin{enumerate}
		\item[5.12]) Prove that there is no largest negative rational number.
		
			\ind $Proof:$
		
			\ind \ind Lean towards the contrary that there is a largest negative
		
			\ind \ind rational number. Let's call that largest negative number $r$,
		
			\ind \ind and let $r \in \Q$. Since $r \in \Q$ , it follows that $r$ can 
		
			\ind \ind be represented as $\frac{-a}{b}$ where $a$ , $b$ $\in \Z$ (by definition). 
		
			\ind \ind Now let's divide $\frac{-a}{b}$ by 2, resulting in
			          $\frac{-a}{2b}$.
		          
			\ind \ind Since $-a$,$2b$ $\in \Z$ and $\frac{-a}{b}$ $\textless$ $\frac{-a}{2b}$, 
			          we have reached
			          
			\ind \ind a contradiction, thus proving there is no largest negative 
		
			\ind  \ind rational number.		
			\ind \ind \ind \ind \ind \ind \ind{%
				\setlength{\fboxsep}{1pt}%
				\setlength{\fboxrule}{2pt}%
				\fbox{\includegraphics[height=.9cm]{proof.jpeg}}
			}%	
			
			
		\item[5.20]) Let $a$ be an irrational number and $r$ be a nonzero rational number. Prove that if $s$ is
		             a real number, then either $ar$+$s$ or $ar$-$s$ is irrational.
		
			\ind $Proof:$
		
			\ind \ind Lean towards the contrary that either $ar$+$s$ or $ar$-.$s$
		
			\ind \ind  is rational. There are 3 cases: Both $ar$+$s$ and $ar$-.$s$ are
		
			\ind \ind rational (Case 1); only $ar$+$s$ is rational (Case 2);  
		
			\ind \ind only $ar$-$s$ is rational (Case 3).
		
			\ind \ind \ind CASE 1:
		          
			\ind \ind \ind By definiton, if $ar$+$s$ and $ar$-.$s$ are rational, then
			          
			\ind \ind \ind $ar$+$s$ and $ar$-.$s$ can be represent as $\frac{p}{q}$ and $\frac{v}{w}$
		
			\ind  \ind \ind respectfully, such that $p,q,v,w \in \Z$.
			
			\ind \ind \ind It  MUST follow that:
			
			\ind \ind \ind \ind $ar + s = \frac{p}{q}$ and $ar - s = \frac{v}{w}$.   
						
			\ind \ind \ind $\iff$ $ar = \frac{p}{q} - s$ and $ar = \frac{v}{w} + s $.   
			
			\ind \ind \ind $\iff$ $ar = \frac{p}{q} - \frac{sq}{q}$ and $ar = \frac{v}{w} + \frac{sw}{w} $.  
			
			\ind \ind \ind $ar = \frac{p - sq}{q} $ and $ar = \frac{v + sw}{w} $.
			
			\ind \ind \ind $a = \frac{p - sq}{rq} $ and $a = \frac{v + sw}{rw} $.  
			
			\ind \ind \ind We have reached a contradiction. Recall in the defintion 
			
			\ind \ind \ind of $a$, we claimed that $a$ was irrational, but since 
			
			\ind \ind \ind $\frac{p - sq}{rq} $,$\frac{v + sw}{rw} \in \Z$, $a$ is not irrational.
			
			\ind \ind \ind This concludes case 1.
			
			\ind \ind Cases 2 and 3 can be ommitted because the 
			
			\ind \ind proofs of cases 2 and 3 can be found in Case 1.
			
			\ind \ind By following $ar + s$ and $ar - s$ seperately, these
			
			\ind \ind prove case 2 and 3 respectfully.
			
			\ind \ind \ind \ind \ind \ind \ind \ind \ind \ind \ind \ind {%
				\setlength{\fboxsep}{1pt}%
				\setlength{\fboxrule}{2pt}%
				\fbox{\includegraphics[height=.9cm]{proof.jpeg}}
			}%
			
		\item[5.22]) Prove that $\sqrt{2}$ + $\sqrt{3}$ is irrational. 
		
			\ind $Proof:$ I will prove this by the contradiction method.
		
		    \begin{lem} $\sqrt{6}$ is irrational.
			
			\ind For the lemma, lean towards the contrary. Assume 
			
			\ind $\sqrt{6}$ $\in \Q$, which means $\sqrt{6}$ = $\frac{a}{b}$ where 
			
			\ind $a,b$ $\in \Z$. The square of a rational number is still rational, thus
			
			\ind $\frac{a}{b}$ * $\frac{a}{b}$ = $\sqrt{6}$ * $\sqrt{6}$.
			
			\ind $\iff$ $\frac{a^2}{b^2}$ = 6.
			
			\ind $\iff$ $a^2$ = 6$b^2$.
			
			\ind 6$b^2$ = 2(3($b^2$)), and since 3($b^2$) $\in \Z$, it follows
			
			\ind that a must be even (If the square of an even number is even).
			
			\ind Thus $\exists$ $c$ s.t $a=2c$ where $c \in \Z$. Replacing $a$ with $c$ 
			
			\ind gives $(2c)^2$ = $6b^2$. $\iff$ 4$c^2$ = $6b^2$. 
			
			\ind $\iff$ $2c^2$ = $3b^2$. Now we see $3b^2$ must be even since 
			
			\ind $2c^2$ is even.
			
			\ind If both $a$ and $b$ are even, they share atleast a common divisor of 2.
			
			\ind We have reached a contradiction (by the definition of rational  
			
			\ind numbers, $a$ and $b$ must be coprime, which they are not 
			
			\ind if they share 2 as a factor). This ends the lemma, demonstrating
			
			\ind that $\sqrt{6}$ is irrational.
 			 
			\end{lem}
			
			
		
			\ind \ind Assume that $\sqrt{2}$ + $\sqrt{3}$ is rational. Since 
			
			\ind \ind $\sqrt{2}$ + $\sqrt{3}$ $\in \Q$, then the square of $\sqrt{2}$ + $\sqrt{3}$ is
			
			\ind \ind $\in \Q$ as well (this is obvious).
			
			\ind \ind Expressing the square of $\sqrt{2}$ + $\sqrt{3}$ yields $(\sqrt{2} + \sqrt{3})^2$.
			
			\ind \ind $\iff$ 2 + $\sqrt{2} * \sqrt{3} $ + $\sqrt{2} * \sqrt{3} $ + 3.
			
			\ind \ind $\iff$ 5 + 2$\sqrt{6}$.
			
			\ind \ind We have reached a contradiction. If $\sqrt{2}$ + $\sqrt{3}$
			
			\ind \ind was rational, the result of the square would be as well. But 
			
			\ind \ind since $\sqrt{6}$ is irrational, by the lemma, then it is impossible 
			
			\ind \ind $\sqrt{2}$ + $\sqrt{3}$ is rational. (2 times an irrational number
			
			\ind \ind  is still irrational).
			
			\ind \ind \ind \ind \ind \ind \ind \ind \ind \ind \ind \ind {%
				\setlength{\fboxsep}{1pt}%
				\setlength{\fboxrule}{2pt}%
				\fbox{\includegraphics[height=.9cm]{proof.jpeg}}
			}%
			
		\item[5.46] ) Prove there exists a unique real number solution to the equation $x^3 + x^2 -1 = 0$, let's 		
		call this f(x), between $x=\frac{2}{3}$ and $x=1$.
		
		\ind $Proof:$
		
			\ind \ind I will prove this using the Intermediate Value Theorem.
		
			\ind \ind Plugging in $\frac{2}{3}$ into the equation, we get :
			
			\ind \ind $\frac{2^3}{3^3}$ + $\frac{2^2}{3^2}$ - 1.
		
			\ind \ind $\iff$ $\frac{8}{27} + \frac{4}{9}$ - 1.   
		
			\ind \ind $\iff$ $\frac{8}{27} + \frac{12}{27}$ - 1.
			
			\ind \ind $\iff$ $\frac{20}{27}$ - $\frac{27}{27}$.
			
			\ind \ind  $\iff$ $\frac{-7}{27}$. 
			
			\ind \ind Now plugging in 1 in the equation, we get :
			
			\ind \ind $1^3$ + $1^2$ -1.
			
			\ind \ind $\iff$ 1 + 1 - 1.
			
			\ind \ind $\iff$ 1.
			
			Because 	$x^3 + x^2 -1$ is continous on the interval [$\frac{2}{3}$, 1]
			
			($\frac{d}{dx}$[$x^3 + x^2 -1$]=$x^2 + 2x$, which is strictly increasing), and zero
			
			is between the outputs of f($\frac{2}{3}$) and f(1), by the I.V.T there must be a real number
			$c$ s.t. f($c$) = 0. This proves a real number solution exists.
			
			Let $r$ $\in$ [$\frac{2}{3}$, 1] and f($r$)=0.
			
			It's the case such that : $r^3 + r^2 -1 = 0 = c^3 + c^2 - 1$.
			
			$\iff r^3 - c^3 + r^2 - c^2 = 0$.			
			
			$\iff (r-c)(r^2 + rc + c^2)+(r-c)(r+c) = 0$.
			
			$\iff (r-c)[(r^2 + rc + c^2)+(r+c)] = 0$.
			
			$\iff (r-c) = 0$.
			
			$\iff r = c$.
			
			This proves the real solution is unique.
			
			\ind \ind \ind \ind \ind \ind \ind \ind \ind \ind \ind \ind {%
				\setlength{\fboxsep}{1pt}%
				\setlength{\fboxrule}{2pt}%
				\fbox{\includegraphics[height=.9cm]{proof.jpeg}}
			}%		
	\end{enumerate}

\end {document}