\documentclass[12pt]{article}

\usepackage{amsfonts}
\usepackage{amsmath}
\usepackage{amssymb}
\usepackage{amsthm}

\newcommand{\dlim}{\displaystyle\lim}
\newcommand{\theorem}{{\bf Theorem.}}

\pagestyle{empty}

\newcommand{\Q}{\mathbb Q}
\newcommand{\N}{\mathbb N}
\newcommand{\R}{\mathbb R}

\begin{document}

\noindent
{\bf\large Using Align and Align*}

\bigskip

This is the align environment. 

\begin{align}
(x-3)(x-5)+5 & = x^2-8x+15+5\\
& = x^2-8x+20\\
& = (x-4)^2+4\\
& \geq 4.
\end{align}

This is the align* environment.

\begin{align*}
(x-3)(x-5)+5 & = x^2-8x+15+5\\
& = x^2-8x+20\\
& = (x-4)^2+4\\
& \geq 4.
\end{align*}

Both environments automatically place every entry into math mode, so you do not need to use dollar signs.

Using additional ampersands, here I've created a third column into which I've inserted comments. I used \textbackslash text to make the comments appear as regular text within the math mode environment. I do not know why the text is right justified.

\begin{align*}
(x-3)(x-5)+5 & = x^2-8x+15+5 & \text{Expand} \\
& = x^2-8x+20 & \text{Combine constants}\\
& = (x-4)^2+4 & \text{I am completing the square}\\
& \geq 4. & \text{Because $(x-4)^2\geq 0$}
\end{align*}
\end{document} 