\documentclass[12pt]{article}

% These are some useful packages
\usepackage{amsfonts}
\usepackage{amsmath}
\usepackage{amssymb}
\usepackage{amsthm}

% This allows me to type nice looking symbols for the rational, natural, integer, and real numbers using only two keystrokes
\newcommand{\Q}{\mathbb Q}
\newcommand{\N}{\mathbb N}
\newcommand{\Z}{\mathbb Z}
\newcommand{\R}{\mathbb R}

%This creates an environment for a lemma
\newtheorem{lemma}{Lemma}


\begin{document}

\title{Some \LaTeX Examples}
\author{D. Seabold}

\maketitle

\noindent
Common symbols: $\in$ $\notin$ $\subset$ $\subseteq$ $\nsubseteq$ $\cup$ $\cap$ $\bigcup$ $\bigcap$ $\times$ $\mid$ $\equiv$ $\not\equiv$ $\vee$ $\wedge$ $\implies$ $\iff$ $\longrightarrow$ $\circ$

\begin{enumerate}
\item Given any two real numbers $x$ and $y$ with $x<y$, there is a real number $z$ such that $x<z<y$.

\item  If $a, b\in\N$ and $b\neq 0$, then $\frac{a}{b} \in \Q$.
        
\item My fraction doesn't have to be cramped $(\frac{x^3-3x+1}{x^2-2})$. It can look like $\left(\displaystyle\frac{x^3-3x+1}{x^2-2}\right)$ instead. Check out the code for the large parentheses.

\item For every $x\in \R$, 
    \begin{itemize}
    \item  $x<0$, 
    \item  $x=0$, or
    \item  $x>0$.
    \end{itemize}
    
\item For every $x\in \R$,
    \begin{enumerate}
    \item  $x<0$,
    \item  $x=0$, or
    \item  $x>0$.
    \end{enumerate}
    
\item  For any two sets $A$ and $B$

    $$\overline{A\cup B}=\overline{A}\cap\overline{B}.$$
    
\item  I can also typeset that like this:

    \[ \overline{A\cup B}=\overline{A}\cap\overline{B} \]
    
\item  Suppose $A_1$, $A_2$, $A_3$, are sets. Distinguish between the union of two sets
    
    $$A_1\cup A_2$$
    
    and the union of infinitely many sets:
    
    $$\bigcup_{i=1}^{\infty} A_i.$$   
    
\item Suppose that $A\subseteq B$. Then if $x\notin B$, it follows that $x\notin A$.    
    
\item  If $f:\R\longrightarrow \R$ is differentiable, then it is continuous.
    
\item  If both $f:A\longrightarrow B$ and $g:B\longrightarrow C$ are bijections, then their composition $g\circ f$ and the inverse $f^{-1}$ are bijections as well.
    
\item  If $n\equiv 3 \pmod 4$ then $n^2\equiv 1 \pmod 4$. Of course, this implies that $n^2\not\equiv 2 \pmod{4}$.
    
\item  If $n$ is a prime number and $n\mid ab$ then either $n\mid a$ or $n\mid b$. Hence if $n\nmid a$ then $n\mid b$.

\item  Two sets, $A$ and $B$, are said to be {\em disjoint} if $$A\cap B=\emptyset.$$
    
\item If $0\leq a\leq b$ then $a^2\leq b^2$.

\item  Let $S=\{ n\in\Z : n^2+3n-1\geq 0\}$.

\item  Let $T=\{ 2, 4, 8, 16, \ldots, 1024\}$.

\item  For every $\epsilon>0$ there is a $\delta>0$ such that $|f(x)-f(y)|<\epsilon$ whenever $|x-y|<\delta$.
    

\end{enumerate}

\begin{lemma} If $0<x$ then there exists a $y$ such that $0<y<x$.
\end{lemma}
\begin{proof}
Let $y=x/2$.
\end{proof}

\begin{lemma} For every positive integer $n$, $1\cdot 3\cdot 5\cdots (2n-1)=\displaystyle\frac{(2n)!}{2^n\cdot n!}$.
\end{lemma}
\begin{proof}
Left to the reader. (Hint: Use induction.)
\end{proof}



 \end{document}   